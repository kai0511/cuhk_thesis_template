The recent years have witnessed a rapid advance in machine learning (ML) algorithms and increasing availability of biomedical data. There is great opportunity to integrate the advances both fields to benefit healthcare. One of the long-standing problem in biomedical science is the high cost and high failure rate of drug development. Another related problem is that the same drug or treatment may not have the same effect on every patient, yet it is difficult to precisely estimate the benefit of treatment for each person based on his/her background. 

In this thesis, we seek to address these concerns by application of ML methods on biomedical data, especially "omics" data. In this study, we proposed a computational drug repurposing framework using drug expression profiles, leveraging ML methods. Expression data was used as predictors and drug indication was considered as outcomes. We found that the method can 're-discover' known drugs for treatment, demonstrating its usefulness. 

Drug repurposing however is not always available, and uncovering valid drug targets for specific diseases remain a major challenge. Drug development suffers from high failure rate largely due to the wrong target pursued. To tackle this issue, we proposed a computational framework to identify promising drug targets for further studies, without using knowledge of the known drug targets. We showed that the identified targets from our method were enriched for targets identified from different lines of evidence. 

Precision medicine has been advocated in the past years. One of the major aims is to estimate\textit{ individualized} treatment effects (ITE), i.e. the effect of a treatment (or risk factor) for each person based one's clinical/genetic background. Here we employed forests-based methods to achieve the goal with consideration of each subject's clinical and genetic information. We also proposed an 'imputation' approach to incorporate time-to-event data. Additionally, since there are no well-established methods to evaluate model fitting, we proposed several statistical methods to address this issue. To examine the validity of our framework, we carried out simulation studies, which showed that our proposed statistical methods maintained good type I error control and had strong power in detecting effect heterogeneity. We also applied our approach to GWAS data of COVID-19 to study ITE of clinical variables on the severity of COVID-19, given genetic and other clinical variables as covariates. Taken together, we hope the proposed methods will open new avenues for translating genetic data into clinical applications. 

\newpage
\begin{CJK*}{UTF8}{gbsn}
  近年来,机器学习算法得到了飞速发展,同时生物医学数据的获取也越来越容易。融合这两个领域的进步有利于医疗保健的发展。生物医学发展中长期存在的两个问题是药物开发的高成本和高失败率;另一个相关的问题是,相同的药物或治疗可能会对患者产生不同的效果,然而现在很难依据个人的临床背景准确预测治疗效果。

  在本文中,我们试图通过将机器学习的方法应用于生物医学数据(尤其是生物“组学”数据)来解决上述问题。在这项研究中,我们提出了一种基于药物的基因表达图谱的计算药物重用框架,同时该计算框架基于机器学习算法。在该框架中,药物的基因表达数据被用作自变量,药物适应症作为预测目标。实验结果证实该方法可以“重新发现”已用于治疗的相关疾病的药物,这也证明了该方法的有效性。

  然而,药物重用并不总是可行,同时为特定疾病找到有效的药物靶标仍然是一项重大挑战。药物开发的高失败率很大程度上归咎于靶标选择错误。为了解决这个问题,我们提出了一个不使用已知药物靶标相关信息来寻找药物靶标候选的计算框架来找到有希望的药物靶标用于未来的研究。实验结果证明我们的方法识别出来的药物靶标候选能被不同来源的已知药物靶标印证。
    
  近些年精准医学一直被提倡。精准医学的主要目的之一是预测个人治疗效果,即是对单个人的治疗效果(或危险因素)的预测,同时需要考虑到每个人的临床和者遗传背景。在这里,我们采用基于森林的机器学习的方法来达到这一目的,该方法在预测过程中会考虑到个人的临床和遗传信息。此外,我们还提出了一种基于“估计”方法来处理生存数据。由于目前还没有成熟的对该类模型拟合进行评估的方法,因此我们提出了几种统计方法来解决此问题。为了检验我们框架的有效性,我们进行了仿真研究,仿真结果表明我们提出的统计方法很好的控制了第一类统计学错误,并且在检测疗效的异质性方面具有强大的能力。此外,我们还将我们的方法应用于感染新冠肺炎的病人的全基因组关联研究。在该研究中,新冠肺炎感染严重程度被作为因变量,基因表达和其他临床变量作为协变量,我们研究了新冠感染的个体性差异。综上所述,我们希望所提出的方法能为将遗传数据转化为临床应用开辟新的途径。
\clearpage\end{CJK*}

