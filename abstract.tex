The recent years have witnessed a rapid advance in machine learning (ML) algorithms and increasing availability of biomedical data. There is great opportunity to integrate the advances both fields to benefit healthcare. One of the long-standing problem in biomedical science is the high cost and high failure rate of drug development. Another related problem is that the same drug or treatment may not have the same effect on every patient, yet it is difficult to precisely estimate the benefit of treatment for each person based on his/her background. 

In this thesis, we seek to address these concerns by application of ML methods on biomedical data, especially "omics" data. In this study, we proposed a computational drug repurposing framework using drug expression profiles, leveraging ML methods.  Expression data was used as predictors and drug indication was considered as outcomes. We found that the method can 're-discover' known drugs for treatment, demonstrating its usefulness. 

Drug repurposing however is not always available, and uncovering valid drug targets for specific diseases remain a major challenge. Drug development suffers from high failure rate largely due to the wrong target pursued. To tackle this issue, we proposed a computational framework to identify promising drug targets for further studies, without using knowledge of the known drug targets. We showed that the identified targets from our method were enriched for targets identified from different lines of evidence. 

Precision medicine has been advocated in the past years. One of the major aims is to estimate\textit{ individualized} treatment effects (ITE), i.e. the effect of a treatment (or risk factor) for each person based one's clinical/genetic background. Here we employed tree-based methods to achieve the goal with consideration of each subject's clinical and genetic information. We also proposed an 'imputation' approach to incorporate time-to-event data. Additionally, since there are no well-established methods to evaluate model fitting, we proposed several statistical methods to address this issue. To examine the validity of our framework, we carried out simulation studies, which showed that our proposed statistical methods maintained good type I error control and had strong power in detecting effect heterogeneity. We also applied our approach to GWAS data of COVID-19 to study ITE of clinical variables on the severity of COVID-19, given genetic and other clinical variables as covariates. Taken together, we hope the proposed methods will open new avenues for translating genetic data into clinical applications. 