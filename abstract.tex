There have been a rapid advance in machine learning (ML) and an increasing availability of biomedical data. This offers a great opportunity to integrate them to benefit healthcare. Additionally, drugs and therapies play crucial roles in healthcare. However, drug development is not as successful as anticipated. Meanwhile, doctors still practice the one‐drug‐fits‐all model in clinic. 

In this thesis, we seek to address these concerns by application of ML on biomedical data, especially "omics" data. In this study, we proposed a computational drug repurposing framework based ML methods using drug expression profiles, in which the outcome variable is defined as whether drugs are known to treat the disease. Systematic approaches were utilized to validate our results. 

Drug repurposing is not always available. Moreover, the traditional approach of drug development suffers from high failure rate, largely due to wrong target pursued. Thus, in order to tackle this issue we proposed a computational framework to identify promising drug targets for further study, without the usage of known target information. We verified our results by examining whether our top candidates are more likely to enrich known drug targets.

Precision medicine has been advocated in the past years. Its aim is to estimate individualized treatment effects (ITE). Hence we employed forests based methods to achieve the goal with consideration of subjects' clinical and genetic information, and proposed a "weighted mean" approach to incorporate time-to-event data. Additionally, since there are no well-established methods to evaluate the model fitting, we proposed several statistical methods to address this issue. To examine the validity of our framework in practice, we carried out a simulation study with inclusion of different kinds of relationships across different simulation scenarios. Simulation results show our proposed statistical methods maintain a strong power in detecting the heterogeneity of ITE. We also applied our approach to GAWS data of COVID-19 to study ITE of clinical variables on the severity of COVID-19 infection given genetic and other clinical variables as covariates. Our findings from real data analysis are also supported by previous studies.

we hope that this study will open a new avenue for drug development and estimation of ITE, and hence benefit patients eventually.