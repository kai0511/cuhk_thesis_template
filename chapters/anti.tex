%
% chapter anti-task based loading algorithms


%***************************************************************
\chapter{Anti-Task Based Loading Algorithms}

%------------------------------------
\note{Chapter Outline}
{
This chapter is the beginning of Part II, the study of a new
category of load distribution algorithms.
The ideas behind these new algorithms are two fold.
First, the use of batch assignments which have been proved in Part I 
to be advantageous.
Second, the use of anti-tasks and load state vectors.
This new category of LD algorithms is totally different from
the usual polling-based algorithms, and is first proposed by the writer.

In this chapter, we study the basic version of the new algorithms.
Its benefits and its weaknesses are both evaluated.
In the next two chapters, we discuss further improvements of the algorithms.

\small
The content of this chapter has been published in
\begin{itemize}
\item	``{\it Concurrency: Practice and Experience},''
	10(14):1251--1269, 1998.
\item	``{\it Proceedings, The ? ICDCS},''
\end{itemize}
}
%------------------------------------

Most existing LD algorithms are {\it polling-based\/}.
It means that a node attempting to identify a task transfer partner
will send a query message to a selected target node to ask for its consent
\cite{casavant88,eager86a,eager86b,lu94,lu95b,shiv90,wang85}.
The two negotiating nodes involved may share load information of each other
by explicitly declaring
their load states in the polling and reply messages.
Alternatively, load information can be deduced from the
semantics of messages exchanged.