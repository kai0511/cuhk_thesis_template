% chapter conclusion
\chapter{Conclusion}
  In the thesis, we reviewed the situation of traditional drug development that the number of drugs approved is unproportionate to the increasing amount of investment. This indicates that traditional approaches to drug discovery has not been as successful as anticipated. There is an urgent need of new therapies for some areas, especially for psychiatric disorders, as drug development in psychiatry has become stagnant for some years. Moreover, the traditional approach of drug development is difficult to develop drugs of novel mechanism of actions. On the other hand, the past few years have seen an extremely rapid development in ML methods and applications. Hence computational approaches based on ML methods may be utilized to address this issue, given the wide availability of biomedical data, especially "omics".

  In chapter \ref{chap:Repurposing} we have presented and applied a machine learning to drug repositioning for schizophrenia and depression/anxiety disorders. We found the candidates were enriched for psychiatric drugs considered in clinical trials, and that numerous top hits were supported by previous studies. A systematic literature support analysis showed that the number of article supporting the association between the drug and disease is significantly correlated to predicted treatment probabilities from ML models. The list of repositioning candidates might serve as a useful resource for researchers and clinicians working on schizophrenia as well as depression and anxiety disorders, which are illnesses very much in need of new therapies. In addition, this study may shed light on molecular mechanism of actions of drugs by examining the variable importance provided by forests based methods. On the other hand, our approach may have some spaces for further improvements. In the regard of imbalanced data, more advanced approaches may be employed, such as down-sampling on the class of majority to balance the class size. Meanwhile, increasing the sample size is another direction for further improvement. 
  
  However, drug repositioning may not be alway available in practice. Identifying promising drug target is another direction to hasten drug development, but traditional approach for drug target discovery suffers from high failure rate, mainly due to pursuing wrong targets. In reality, it is impractical to perform in-depth experimental studies on every possible target for each disease. Computational methods offer a cheap, fast and systematic high-throughput approach to guide prioritization of drug targets. Moreover, we have witnessed a rise in the number of studies using computational approaches to discover potential drug targets for further investigation.
  
  In chapter \ref{chap:Target}, we presented a general computational framework to prioritize drug targets for various diseases. Under the framework, different kinds of ML methods can be utilized. We applied four ML methods to identify potential drug target of five disorders. External validation shows that the top candidates are enriched for targets selected by independent lines of evidence from a large external database (Open Targets). Some top target genes were also supported by previous studies. We hope our presented framework can provide an additional way to prioritize drug targets for development, which is independent of and may be combined with other existing sources of data. In addition, a direction for further improvement of this study is that high quality of expression data will dramatically improve the validity of our result, since in practice part of our result is interfered by some potential off-target effects in knock-down experiments. However, this study may provide new path for drug target discovery in future, since it offers a fast, cost-effective way to prioritize  candidates.

  In this thesis, we also explored another hot topic in biomedical studies, personalized medicine, which has been advocated for years. It has been widely acknowledged that patients response differently to a same treatment. This heterogeneity may be caused by their clinical factors and/or genetic backgrounds. Advance in this area can directly benefit to disease treatment in an individual level. The key in personalized medicine is to estimate individualized treatment effects (ITE). However, the traditional golden standard,  randomized control trail (RCT),  is mainly utilized to estimate treatment effects in a population level. Fortunately, we have seen forests based machine learning methods have been successfully employed to estimate ITE. Further, forests based methods can capture high-order interactions naturally presented in biomedical data and enjoy good interpretation in decision-making of model.

  In the chapter \ref{chap:ite}, we proposed a computational framework to estimate ITE of risk factors/treatments on patient outcome. In order to assess model fitting, we proposed different statistical methods to investigate the model fitting. In order to increase the generality of ITE framework, we proposed a weighted "mean imputation" approach to incorporate survival times as outcomes. We carried out simulations to validate our framework on survival outcomes, since if our approach could work in survival data, then it's more readily to accommodate traditional data. Cross all simulation scenarios, our methods maintain a strong power in detecting either constant or heterogeneous treatment effects. In real data analysis, we applied our approach to GWAS data of patients with COVID-19, in which the outcome variable is severity of COVID-19 infection, and experiment results are supported by previous studies. For example, we rediscovered that the cardiometabolic disease is a risk factor to sever COVID-19 infection.

  Our approach may have following limitations. In real data analysis, we found split-half correlation test may show better in detecting the presence of heterogeneity. This contradicts the fact we observed from simulation analysis. Even though some simulation scenarios shed light on the possible explanation that there are lots of covariates of small effects contributing to the outcome in GWAS data analysis, more evidences are needed to clear this concern. 

  In this thesis, we widely studied hot topics ranging from drug development to personalized medicine by computational approaches. This may provide a relatively new angle to explore some existing key issues in medical studies or even in clinical applications, and hence may provide new insights on solutions to these critical issues.
\chapterend
