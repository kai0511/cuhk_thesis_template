% chapter conclusion
\chapter{Conclusions}
  The number of drugs approved in the last few decades is disproportionate to the increasing amount of investment. This indicates that traditional approaches to drug discovery has not been as successful as anticipated. There is an urgent need of new therapies, especially in some areas such as psychiatry. Moreover, drugs of novel mechanism of actions are decreasing in number, suggesting limitations of the traditional drug development approach. On the other hand, the past few years have seen an extremely rapid development in ML methods and applications. Hence computational approaches based on ML methods may be utilized to address this issue, given the wide availability of omics data. 

  In chapter \ref{chap:Repurposing} we have presented and applied a machine learning to drug repositioning for schizophrenia and depression/anxiety disorders. We found the candidates were enriched for psychiatric drugs considered in clinical trials, and that numerous top hits were supported by previous studies. A systematic literature support analysis showed that the number of article supporting the association between the drug and disease is significantly correlated to predicted treatment probabilities from ML models. The list of repositioning candidates might serve as a useful resource for researchers and clinicians working on schizophrenia as well as depression and anxiety disorders, which are illnesses very much in need of new therapies. In addition, this study may shed light on molecular mechanism of actions of drugs by examining the variable importance provided by ML methods. On the other hand, our approach still has room for further improvements. In the regard of imbalanced data, more advanced approaches may be employed, such as down-sampling on the class of majority to balance the class size. Meanwhile, increasing the sample size is another direction for further improvement. 
  
  However, drug repositioning may not be always be available in practice. Identifying promising drug target is another direction to hasten drug development, but traditional approach for drug target discovery suffers from high failure rate. In reality, it is impractical to perform in-depth experimental studies on every possible target for each disease. Computational methods offer a cheap, fast and systematic high-throughput approach to guide prioritization of drug targets. Moreover, we have witnessed a rise in the number of studies using computational approaches to discover potential drug targets for further investigation.

  In chapter \ref{chap:Target}, we presented a general computational framework to prioritize drug targets for various diseases. Under the framework, different kinds of ML methods can be utilized. We applied four ML methods to identify potential drug target of five disorders. External validation shows that the top candidates were enriched for targets selected by independent lines of evidence from a large external database (Open Targets). Some top target genes were also supported by previous studies. We hope our presented framework can provide an additional way to prioritize drug targets for development, which is independent of and may be combined with other existing sources of data. In addition, a direction for further improvement  is that high quality of expression data will dramatically improve the validity of our result, since in practice part of our result is interfered by potential off-target effects in knock-down experiments. 
  
  In this thesis, we also explored the estimation of individualized treatment effects. It has been widely acknowledged that patients response differently to the same treatment. This heterogeneity may be caused by their different clinical and/or genetic backgrounds. Advances in this area can directly benefit disease treatment at an individual level. Here we have employed forest-based machine learning methods  to estimate ITE. Such methods can capture high-order interactions commonly present in biomedical data.

  In the chapter \ref{chap:ite}, we proposed a computational framework to estimate ITE of risk factors/treatments on patient outcome. In order to assess model fitting, we proposed different statistical methods to investigate the model fitting. In order to increase the generality of ITE framework, we proposed a weighted "mean imputation" approach to incorporate survival times as outcomes. We carried out simulations to validate our framework on survival outcomes. We also illustrated the usefulness of this approach by application in real clinical datasets. 

  In this thesis, we studied translational bioinformatic approaches ranging from drug development to estimation of personalized treatment effect. We believe this work will provide a new angle to explore address some key issue in these fields, with the hope to benefit patient care ultimately.  
\chapterend

